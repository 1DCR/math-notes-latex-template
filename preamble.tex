% ============================================================================================ %
%                                                                                              %
%                     @@@@@@@   @@   @@@@@@@@@@@@@        @@@@@@@ @@@@@@                       %
%                        @@@    @@@   @    @@@   @@           @@@   @@                         %
%                        @@@   @@ @@ @@    @@@  @@@@@@@@@@@@   @@@ @@                          %
%                        @@@   @@@@@@      @@@    @@@     @@     @@@                           %
%                        @@@  @     @@     @@@    @@@      @@   @@@@@                          %
%                        @@@ @@@ @@@@@@    @@@    @@@   @      @@  @@@                         %
%                        @@@    @@         @@@    @@@@@@@     @@    @@@                        %
%                      @@@@@@@@@@@      @@@@@@@@  @@@   @  @@@@@@   @@@@@@                     %
%                                                 @@@      @@                                  %
%                                               @@@@@@@@@@@@                                   %
%                                                                                              %
% ============================================================================================ %
%                                                                                              %
%                   < MiKTex + VS Code + LaTex Workshop + LaTeX Utilities >                    %
%                                                                                              %
% Для компиляции требуется движок XeLaTex. Чтобы установить его по-умолчанию + сохранять все   %
% файлы компиляции в отдельную папку "./tmp/" - в settings.json добавить:                      %
%                                                                                              %
%        "latex-workshop.latex.tools": [                                                       %
%            {                                                                                 %
%                "name": "latexmk",                                                            %
%                "command": "latexmk",                                                         %
%                "args": [                                                                     %
%                    "-synctex=1",                                                             %
%                    "-interaction=nonstopmode",                                               %
%                    "-file-line-error",                                                       %
%                    "-xelatex",                                                               %
%                    "-auxdir=./tmp",                                                          %
%                    "-outdir=./",                                                             %
%                    "%DOC%"                                                                   %
%                ],                                                                            %
%                "env": {}                                                                     %
%            }                                                                                 %
%            ],                                                                                %
%            "latex-workshop.latex.recipes": [                                                 %
%                {                                                                             %
%                    "name": "latexmk (xelatex)",                                              %
%                    "tools": ["latexmk"]                                                      %
%                }                                                                             %
%            ],                                                                                %
%            "latex-workshop.latex.recipe.default": "latexmk (xelatex)"                        %
%                                                                                              %
% Если возникают проблемы с polyglossia - нужно его обновить (а лучше - сразу все пакеты).     %
% Также нужно установить пакеты cm-unicode и cm-super.                                         %
%                                                                                              %
% ============================================================================================ %


% Математика
\usepackage{amsmath}
\usepackage{amssymb}
\usepackage{amsthm}
\usepackage{amsfonts}

% Шрифты и языки
\usepackage{fontspec}
\usepackage{polyglossia}

\setmainfont{CMU Serif}
\setsansfont{CMU Sans Serif}
\setmonofont{CMU Typewriter Text}

\setmainlanguage{russian}
\setotherlanguage{english}

% Геометрия страницы
\usepackage{geometry}
\geometry{
 a4paper,
 headheight=15pt,
 total={170mm,247mm},
 left=25mm,
 top=25mm
}

% Рисунки
\usepackage{graphicx}
\graphicspath{ {./figures/} }

% Подписи
\usepackage[margin=5pt,font={small, singlespacing}, labelfont={small}, justification=centering, labelsep=period]{caption}
\captionsetup{belowskip=0pt}

% Настройка заголовков
\usepackage{titlesec}

\titleformat{\section}
    {\normalfont\Large\bfseries\centering}
    {\thesection.}
    {1em}
    {}

% Остальные пакеты
\usepackage{array}
\usepackage{enumerate}
\usepackage{float}
\usepackage{hyperref}
\usepackage{listings}
\usepackage{multicol}
\usepackage{xcolor}

% Оформление теорем, определений, лемм и доказательств
\usepackage[framemethod=TikZ]{mdframed}

% Теоремы
\newcounter{thm}[section]
\setcounter{thm}{0}
\renewcommand{\thethm}{\arabic{section}.\arabic{thm}}

\newenvironment{thm}[2][]{
    \refstepcounter{thm}
    \ifstrempty{#1}{
        \mdfsetup{
            frametitle={
                \tikz[baseline=(current bounding box.east),outer sep=0pt]
                \node[draw=gray!50, line width=1pt, anchor=east, rectangle, fill=white]
                {\strut Теорема~\thethm};
            }
        }
    }{
        \mdfsetup{
            frametitle={
                \tikz[baseline=(current bounding box.east),outer sep=0pt]
                \node[draw=gray!50, line width=1pt, anchor=east, rectangle, fill=white]
                {\strut Теорема~\thethm:~#1};
            }
        }
    }
    \mdfsetup{
        innertopmargin=8pt,
        linecolor=gray!50,
        linewidth=1pt,
        topline=true,
        frametitleaboveskip=\dimexpr-\ht\strutbox\relax,
        backgroundcolor=none
    }
    \begin{mdframed}[]\relax
    \label{#2}
}{\end{mdframed}}

% Определения
\newcounter{dfn}[section]
\setcounter{dfn}{0}
\renewcommand{\thedfn}{\arabic{section}.\arabic{dfn}}

\newenvironment{dfn}[2][]{
    \refstepcounter{dfn}
    \ifstrempty{#1}{
        \mdfsetup{
            frametitle={
                \tikz[baseline=(current bounding box.east),outer sep=0pt]
                \node[draw=gray!50, line width=1pt, anchor=east, rectangle, fill=white]
                {\strut Определение~\thedfn};
            }
        }
    }{
        \mdfsetup{
            frametitle={
                \tikz[baseline=(current bounding box.east),outer sep=0pt]
                \node[draw=gray!50, line width=1pt, anchor=east, rectangle, fill=white]
                {\strut Определение~\thedfn:~#1};
            }
        }
    }
    \mdfsetup{
        innertopmargin=8pt,
        linecolor=gray!50,
        linewidth=1pt,
        topline=true,
        frametitleaboveskip=\dimexpr-\ht\strutbox\relax,
        backgroundcolor=none
    }
    \begin{mdframed}[]\relax
    \label{#2}
}{\end{mdframed}}

% Леммы
\newcounter{lem}[section]
\setcounter{lem}{0}
\renewcommand{\thelem}{\arabic{section}.\arabic{lem}}

\newenvironment{lem}[2][]{
    \refstepcounter{lem}
    \ifstrempty{#1}{
        \mdfsetup{
            frametitle={
                \tikz[baseline=(current bounding box.east),outer sep=0pt]
                \node[draw=gray!50, line width=1pt, anchor=east, rectangle, fill=white]
                {\strut Лемма~\thelem};
            }
        }
    }{
        \mdfsetup{
            frametitle={
                \tikz[baseline=(current bounding box.east),outer sep=0pt]
                \node[draw=gray!50, line width=1pt, anchor=east, rectangle, fill=white]
                {\strut Лемма~\thelem:~#1};
            }
        }
    }
    \mdfsetup{
        innertopmargin=8pt,
        linecolor=gray!50,
        linewidth=1pt,
        topline=true,
        frametitleaboveskip=\dimexpr-\ht\strutbox\relax,
        backgroundcolor=none
    }
    \begin{mdframed}[]\relax
    \label{#2}
}{\end{mdframed}}

% Доказательства
\newcounter{prf}[section]
\setcounter{prf}{0}
\renewcommand{\theprf}{\arabic{section}.\arabic{prf}}

\newenvironment{prf}[2][]{
    \refstepcounter{prf}
    \ifstrempty{#1}{
        \mdfsetup{
            frametitle={
                \tikz[baseline=(current bounding box.east),outer sep=0pt]
                \node[draw=gray!50, line width=1pt, anchor=east, rectangle, fill=white]
                {\strut Доказательство~\theprf};
            }
        }
    }{
        \mdfsetup{
            frametitle={
                \tikz[baseline=(current bounding box.east),outer sep=0pt]
                \node[draw=gray!50, line width=1pt, anchor=east, rectangle, fill=white]
                {\strut Доказательство~\theprf:~#1};
            }
        }
    }
    \mdfsetup{
        innertopmargin=8pt,
        linecolor=gray!50,
        linewidth=1pt,
        topline=true,
        frametitleaboveskip=\dimexpr-\ht\strutbox\relax,
        backgroundcolor=none
    }
    \begin{mdframed}[]\relax
    \label{#2}
}{\end{mdframed}}

